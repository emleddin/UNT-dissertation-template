%%%%%%%%%%%%%%%%%%%%%%%%%%%%%%%%%%%%%%
\chapter{IT'S EASY AS 1 2 3}
%%%%%%%%%%%%%%%%%%%%%%%%%%%%%%%%%%%%%%

In this chapter root systems and Chevalley bases for specific matrix
representations of some of the classical, simple, complex Lie algebras are
constructed. Each classical, simple, complex Lie algebra is a Lie subalgebra
of $\fgl_m(\BBC)$ for some $m$. The subalgebra of diagonal matrices in such
a Lie algebra will be denoted by $H$. It turns out that for the matrix
representations considered, $H$ is a maximal toral subalgebra.

It's possible to refer to displayed equations at any time, even before they
appear. For example \cref{eq:bigmat} is defined somewere below. This is not
good form, but sometimes it's handy.

For positive integers $i$, $j$, and $n$ with $1\leq i,j\leq n$, let
$e_{i,j}$ denotes the square matrix whose only non-zero entry is a $1$ in
row $i$ and column $j$.  Denote the $n\times n$ diagonal matrix with entries
$a_1$, \dots, $a_n$ by $\diag{ \seq a}$. Then
\[
\diag{ \seq a}=
\begin{bmatrix}
  a_1&0&&&\dots&0 \\ 0&a_2&&&\dots&0 \\
  \vdots&&\ddots&&&\vdots \\ 0&&\dots&&a_{n-1}& \\ 0&&\dots&&0 &a_n
\end{bmatrix}
= \sum_{i=1}^n a_i e_{i,i}.
\]

Clearly the set $\{\, e_{i,i}\mid i\leq i\leq n\,\}$ is a basis of the
vector space of diagonal matrices. Suppose $h=\diag{\seq a}$ and $1\leq
i,j\leq n$, then
\begin{align*} 
  [h, e_{i,j}] &= he_{i,j} -e_{i,j} h\\
  &= \sum_{k=1}^n a_k\,e_{k,k} e_{i,j} -\sum_{k=1}^n a_k\, e_{i,j}e_{k,k}\\
  &= \sum_{k=1}^n a_k\, \delta_{k,i}e_{k,j} - \sum_{k=1}^n \, a_k
  \delta_{j,k} e_{i,k} \\
  &= a_i\, e_{i,j} - a_j\, e_{i,j}\\
  &= (a_i-a_j)\, e_{i,j}.
\end{align*}

%%%%%%%%%%%%%%%%%%%%%%%%%%%%%%%%%%%%%%%%%%%%%%%%%%%%%%%%%%%%%%%%%%%%
\section{Type \texorpdfstring{$B_n$}{Bn}, the Odd-dimensional, Orthogonal
  Lie Algebras}

The odd-dimensional, orthogonal Lie algebra $\fso_{2n+1}(\BBC)$, or simply
$\fso_{2n+1}$, is the set of all matrices $X$ in $\fgl_{2n}(\BBC)$ such that
\[
JX= -X^t J
\]
where $J= \left[\begin{smallmatrix} 1&0&0\\ 0 &0&I_n\\ 0&I_n
    &0\end{smallmatrix} \right]$.  

Suppose $X=\left[\begin{smallmatrix} a&s&t\\ u&A&B\\ v&C&
    D \end{smallmatrix} \right]$, where $a$ is a complex number, $s,t,u,v$
are vectors with $n$ components, and $A, B, C, D$ are $n\times n$
matrices. Then $JX= -X^t J$ if and only if
\[
\begin{bmatrix}
  1&0&0\\ 0&0&I_n\\ 0&I_n& 0
\end{bmatrix}
\begin{bmatrix}
  a&s&t\\ u&A&B\\ v&C& D
\end{bmatrix}
=
\begin{bmatrix}
  -a&-u^t&-v^t\\-s^t&-A^t&-C^t\\ -t^t&-B^t&-D^t
\end{bmatrix}
\begin{bmatrix}
  1&0&0\\ 0&0&I_n\\ 0&I_n& 0
\end{bmatrix},
\]
which is if and only if
\[
a=0\quad u=-t^t \quad v=-s^t\quad D=-A^t,\quad B=-B^t,\quad\text{and} \quad
C=-C^t.
\]
If the $i\th$ entry of $s$ is $s_i$, $i\th$ entry of $t$ is $t_i$, the
$(i,j)$ entry of $A$ is $a_{i,j}$, the $(i,j)$ entry of $B$ is $b_{i,j}$,
and the $(i,j)$ entry of $C$ is $c_{i,j}$, then $\left[\begin{smallmatrix}
    a&s&t\\ u&A&B\\ v&C& D \end{smallmatrix} \right]$ is in $\fso_{2n+1}$ if
and only if
\begin{equation}
  \label{eq:bigmat}
  \begin{bmatrix}
    a&s&t\\ u&A&B\\ v&C& D
  \end{bmatrix}
  =
  \begin{bmatrix}
    0&s_1&s_2&\dots&s_n &t_1&t_2&\dots&t_n\\
    -t_1& a_{1,1}&a_{1,2}&\dots&a_{1,n}&0&b_{1,2}&\dots&b_{1,n} \\
    -t_2& a_{2,1}&a_{2,2}&\dots&a_{2,n}&-b_{2,1}&0&\dots&b_{2,n} \\
    \vdots&\vdots&\vdots&\ddots&\vdots&\vdots&\vdots&\ddots&\vdots \\
    -t_n& a_{n,1}&a_{n,2}&\dots&a_{n,n}&-b_{n,1}&-b_{n,2}&\dots&0 \\
    -s_1& 0&c_{1,2}&\dots&c_{1,n}&-a_{1,1}&-a_{n,2}&\dots&-a_{n,1} \\
    -s_2& -c_{2,1}&0&\dots&c_{1,n}&-a_{2,1}&-a_{2,2}&\dots&-a_{2,n} \\
    \vdots & \vdots&\vdots&\ddots&\vdots&\vdots&\vdots&\ddots&\vdots \\
    -s_n& -c_{n,1}&-c_{n,2}&\dots&0&-a_{1,n}&-a_{n,2}&\dots&-a_{n,n}
  \end{bmatrix}.
\end{equation}

For $1\leq i\leq n$, define $x_i$ in $H^*$ by
\[
x_i(h)= a_i\quad\text{when}\quad h=\diag{0,\seq a, \seq{-a}}.
\]
Then $x_i(h)$ is the coefficient of $d_i$ when $h$ is expressed as a linear
combination of vectors in $\CB_H$.

The set
\begin{multline*}
  \CB=\CB_H \cup \{\, e_{1, j+1}-e_{n+j+1,1}\mid 1\leq j\leq n\,\} \cup \{\,
  e_{1, n+j+1}-e_{j+1,1}\mid 1\leq j\leq n\,\} \\
  \cup \{\, e_{i+1,j+1} -e_{n+j+1, n+i+1} \mid 1\leq i\ne j\leq n\,\} \cup
  \{\, e_{i+1,n+j+1} -e_{j+1, n+i+1} \mid 1\leq i< j\leq n\,\}  \\
  \cup \{\, e_{n+i+1,j+1} -e_{n+j+1, i+1}\mid 1\leq i< j\leq n\,\}
\end{multline*}
is a basis of $\fso_{2n+1}$. In particular, $\dim \fso_{2n+1}=
3n+n^2-n+2\binom n2= 2n^2+n$.

\begin{proposition}
  The set $\CB\setminus \CB_H$ consists of root vectors.
\end{proposition}

\begin{proof}
  This is proved by direct computation. There are five cases.

  Suppose that $h=\diag{0,\seq a, \seq{-a}}$ is in $H$.

  Consider $e_{1, j+1}-e_{n+j+1,1}$ where $1\leq j\leq n$. Then
  \begin{align*}
    [h, e_{1, j+1}-e_{n+j+1,1}]& =[h,e_{1,j+1}] -[h,e_{n+j+1, 1}]\\
    &= -a_j\, e_{1,j+1} -a_{j} \, e_{n+j+1, 1} \\
    &= (-a_j)\, (e_{1, j+1}-e_{n+j+1,1}) \\
    &= (-x_j)(h) \, (e_{1, j+1}-e_{n+j+1,1}).
  \end{align*}
  Thus, $e_{1, j+1}-e_{n+j+1,1}$ is a root vector. The corresponding root is
  the linear function $-x_j$ in $H^*$.

  Consider $e_{i+1,j+1} -e_{n+j+1, n+i+1}$ where $1\leq i\ne j\leq n$. Then
  \begin{align*}
    [h, e_{i+1,j+1} -e_{n+j+1, n+i+1}]& =[h,e_{i+1,j+1}] -[h,e_{n+j+1,
      n+i+1}]\\
    &= (a_i-a_j)\, e_{i+1,j+1} - (-a_{j} + a_{i}) \, e_{n+j+1, n+i+1} \\
    &= (a_i-a_j)\, (e_{i+1,j+1} -e_{n+j+1, n+i+1}) \\
    &= (x_i-x_j)(h) \, (e_{i+1,j+1} -e_{n+j+1, n+i+1}).
  \end{align*}
  Thus, $e_{i+1,j+1} -e_{n+j+1, n+i+1}$ is a root vector. The corresponding
  root is the linear function $x_i-x_j$ in $H^*$.

  The other two cases are similar: $e_{1,n+j+1} -e_{j+1, 1}$ is a root
  vector and the corresponding root is the linear function $x_j$ in $H^*$;
  $e_{n+i+1,j+1} -e_{j+1, n+i+1}$ is a root vector and the corresponding
  root is the linear function $-x_i-x_j$ in $H^*$.

  The computations above are summarized in \cref{tab:so} and
  \cref{tab:soodd}. 

  This table is typeset as a float that is strongly encouraged to come next
  if at all possible.
\end{proof}


\begin{table}[h!tb]
  \small \arraycolsep10pt
  \renewcommand{\arraystretch}{1.3}
  \begin{equation*}
    \begin{array}{l|cc}
      i,\ j &\alpha &e_\alpha\\\hline
      1\leq j\leq n&-x_j & e_{1,j+1} -e_{n+j+1, 1}\\
      1\leq j\leq n&x_j & e_{1,n+j+1} -e_{j+1, 1}\\
      1\leq i\ne j\leq n&x_i-x_j &  e_{i+1,j+1} -e_{n+j+1, n+i+1} \\
      1\leq i< j\leq n&x_i+x_j & e_{i+1,n+j+1} -e_{j+1, n+i+1} \\ 
      1\leq i< j\leq n&-x_i-x_j & e_{n+i+1,j+1} -e_{j+1, n+i+1}
    \end{array}  
  \end{equation*}
  \caption{An ad hoc table not set as a float}\label{tab:so}
\end{table}


\renewcommand{\arraystretch}{1.1}
\begin{table}[h!tb]
  \centering
  \begin{tabular} 
    {>{$}c<{$} | >{$}c<{$} >{$}c<{$} >{$}c<{$} >{$}c<{$}}
    \toprule   
    Z_X & C_X' & \CA & C_X & \exp\\ 
    \midrule
%%
    A_0 & G_{33} & Y & Y & Y\\
    A_1 & D_4 & N & N & N\\
    A_1^2 & B_3 & N & Y & N\\
    A_2 & G_{3,1,2} & N & N & N\\
    A_1^3 & G_{6,3,2}& Y & Y & Y\\
    G_{33} & A_0 & Y & Y & Y\\
    \bottomrule
  \end{tabular}
  \caption{A professional table set as a float} \label{tab:soodd}
\end{table}

Here is another reference to \cref{tab:soodd}.

\begin{corollary}
  The the subalgebra $H$ is a maximal toral subalgebra and the root system
  of $(\fso_{2n+1}, H)$ is
  \[
  \Phi=\{\pm(x_i\pm x_j) \mid 1\leq i<j\leq n\,\} \cup \{\, 2x_i\mid 1\leq
  i\leq n\,\}.
  \]
\end{corollary}

\begin{proof}
  By the proposition, $\fso_{2n+1}$ has a root space decomposition. Suppose
  that $H'$ is a toral subalgebra containing $H$. Just suppose that $H'$
  properly contains $H$. Then $H'$ is abelian and there is an element $h'$
  in $H$ that is a linear combination of the basis elements in $\CB\setminus
  \CB_H$. Write $h'=v_\alpha+h''$ where $v_\alpha$ is a non-zero vector in
  the $\alpha$ root space. Then $v_\alpha$ is a non-zero multiple of the
  root vector $e_\alpha$ in $\CB\setminus \CB_H$. Fix $h$ in $H$ such that
  $h$ is not in $\ker \alpha$, then $[h,h']= [h, v_\alpha+h'']=
  \alpha(h)v_\alpha + [h,h'']$. Then $\alpha(h)v_\alpha\ne 0$ and $[h, h'']$
  is in the span of $\CB\setminus(\CB_H\cup \{e_\alpha\})$. Therefore,
  $[h,h'] \ne0$. This contradicts the fact that $H'$ is abelian. Thus,
  $H'=H$ and so $H$ is maximal.
\end{proof}

For $1\leq i\leq n$ define $\alpha_i$ in $H^*$ by
\begin{align*}
  \alpha_i&=x_i-x_{i+1} \quad (1\leq i\leq n-1),\\
  \alpha_n&= x_n.
\end{align*}
Set $\Pi=\{\, \alpha_i\mid 1\leq i\leq n\,\}$. It's easy to see that $\Pi$
is a basis of $H^*$. Notice that the roots $x_i-x_j$ with $i\ne j$ from
\cref{tab:soodd} are split into two subsets depending on whether or not
$i<j$.


%%%%%%%%%%%%%%%%%%%%%%%%%%%%%%%%%%%%%%%%%%%%%%%%%%%%%%%%%%%%%%%%%%%%
\section{Adding a Section to Have More Sections}
The usual Euclidean metric on $H^*$ is defined by 
\[
d(\sum_{i=1}^n a_i x_i, \sum_{i=1}^n b_i x_i) = \sqrt{ \sum_{i=1}^n
  |a_i-b_i|^2}.
\]
With respect to this metric, the roots $\pm (x_i\pm x_j)$ with $i\ne j$ have
length $\sqrt 2$ and the roots $\pm x_i$ have length $1$. Thus, there are
two root lengths. Roots with minimum length are called \emph{short roots}
and roots with maximum length are called \emph{long roots.} The highest root
is a long root.

By direct inspection, there is a unique highest short root,
$x_1=\alpha_1+\dots +\alpha_n$, with height $n$.

Notice that if $\alpha=\sum_{i=1}^n m_i\alpha_i$, then the coefficients
$m_i$ are either all non-negative or all non-positive. Define
\[
\Phi^+=\Big\{\, \alpha=\sum_{i=1}^n m_i\alpha_i\mid mi_\geq 0 \, \forall\,
1\leq i\leq n\,\Big\}
\]
and
\[
\Phi^- = \Big\{\, \alpha=\sum_{i=1}^n m_i\alpha_i\mid mi_\leq 0 \, \forall\,
1\leq i\leq n\,\Big\}.
\]
Then $\Phi^-= -\Phi^+$ and $\Phi = \Phi^+ \coprod \Phi^-$.

We next compute the elements $t_{\alpha_i}$ in $H$ for $1\leq i\leq
n$. Using the basis $\CB$ of $\fso_{2n+1}$ it is straightforward to compute
the restriction of the Killing form to $H$ by computing the matrices of $\ad
h$ and $\ad h'$, and then $\trace(\ad h\circ \ad h')$ for $h$ and $h'$ in
$H$. The result is
\[
\kappa(h,h')= \sum_{\alpha\in \Phi} \alpha(h) \alpha(h').
\]
If $h=\diag{0,\seq a, \seq{-a}}$ and $h'=\diag{0,\seq{a'}, \seq{-a'}}$, then
$\alpha(h) \alpha(h')$ is given in \cref{tab:sodhh'}.
\begin{table}[h!tb]
  \small \arraycolsep10pt
  \renewcommand{\arraystretch}{1.3}
  \begin{equation*}
    \begin{array}{c|c}
      \alpha&\alpha(h) \alpha(h')\\\hline
      x_i-x_j& (a_i-a_j)(a_i'-a_j')\\
      x_i&(a_i)(a_i')= a_ia_i'\\ 
      x_i+x_j&(a_i+a_j)(a_i'+a_j')\\ 
      - x_i+x_j&(-a_i+a_j)(-a_i'+a_j')=(a_i-a_j)(a_i'-a_j')\\
      -x_i &(-a_i)(-a_i')=a_ia_i'\\ 
      -x_i-x_j&=(-a_i-a_j)(-a_i'-a_j')=(a_i+a_j)(a_i'+a_j') 
    \end{array}  
  \end{equation*}
  \caption{$\alpha(h) \alpha(h')$ when $h=\diag{\seq a}$ and
    $h'=\diag{\seq{a'}}$}\label{tab:sodhh'}  
\end{table}

The Killing form $\kappa(h,h')$ can be computed in terms of the coefficients
of $h$ and $h'$ when $h$ and $h'$ are expressed as linear combinations of
$\{x_1, \dots, x_n\}$ as follows.
\begin{equation}
  \label{eq:sod1}
  \begin{aligned}
    \kappa(h,h')&= \sum_{1\leq i<j\leq n} 2\left( (a_i-a_j)(a_i'-a_j') +
      (a_i+a_j)(a_i'+a_j') \right) +2\sum_{i=1}^n \left( a_ia_i' \right)\\
    &= \sum_{1\leq i<j \leq n} (4a_ia_i' +4a_ja_j') +\sum_{i=1}^n 2a_ia_i'
    \\
    &= \sum_{i=1}^n a_ia_i' (2+4(n-i) +4(i-1))\\
    &=(4n-2) \sum_{i=1}^n a_ia_i'.
  \end{aligned}
\end{equation}
The penultimate equality in \cref{eq:sod1} is most easily seen by arranging
the summands in an $n\times n$ array.
\[
%\small
\begin{array}{cccrrrrc}
  2a_1a_1'&4a_1a_1'+ 4a_2a_2'&4a_1a_1'+
  4a_3a_3'&&\dots&\dots&\dots&4a_{1}a_{1}'+ 4a_na_n' \\  
  &2a_2a_2'&4a_2a_2'+4a_3a_3'&&&&&4a_{2}a_{2}'+ 4a_na_n' \\
  &&2a_3a_3'&&&&&4a_{3}a_{3}'+ 4a_na_n' \\
  &&&&&&&\vdots\hphantom{+4a_na_n'}\\
  &&&&\ddots&&&\vdots\hphantom{+4a_na_n'}\\
  &&&&&&&4a_{n-2}a_{n-2}'+ 4a_na_n' \\ 
%  &&&&&&&4a_{n-1}a_{n-1}'+ 4a_na_n' \\
  &&&&&&&2a_na_n'
\end{array}
\]


%%%%%%%%%%%%%%%%%%%%%%%%%%%%%%%%%%%%%%%%%%%%%%%%%%%%%%%%%%%%%%%%%%%%
\section{Type \texorpdfstring{$D_n$}{Dn}: Even dimensional, Orthogonal Lie
  Algebras} 

For $1\leq i\leq n$. Then the element $t_{\alpha_i}$ in $H$ is defined by
the condition that
\[
\kappa(h, t_{\alpha_i}) = \alpha_i(h) \quad \text{for all $h$ in $H$.}
\]

Fix $1\leq i\leq n-1$ and suppose $t_{\alpha_i}= \diag{0,\seq t,
  \seq{-t}}$. Then
\begin{equation*}
  a_i-a_{i+1}= (4n-2)(a_1t_1+\dots+a_it_i+a_{i+1}t_{i+1}+ \dots +a_nt_n)
\end{equation*}
when $h=\diag{0,\seq a, \seq{-a}}$. Thus, $t_1$, \dots, $t_n$ are such that
\begin{equation*}
  \textstyle a_1t_1+\dots+a_i\big(t_i- \frac 1{4n-2} \big) +a_{i+1}
  \big(t_{i+1} +\frac 1{4n-2} \big)+  \dots +a_nt_n=0 
\end{equation*}
for all $a_1$, \dots, $a_n$ in $\BBC$. Taking $a_j=1$ and $a_k=0$ for $k\ne
j$ we see that
\[
t_j= \begin{cases} \textstyle \frac 1{4n-2}&j=i \\ -\frac 1{4n-2} &j=i+1 \\0
  &j\ne i, i+1. \end{cases}
\]
Therefore, for $1\leq i\leq n-1$, $t_{\alpha_i}=\frac 1{4n-2}( d_i-
d_{i+1})$.

Now consider $t_{\alpha_n}$. Say $t_{\alpha_i}= \diag{0,\seq t,
  \seq{-t}}$. Then
\begin{equation*}
  a_n= (4n-2)(a_1t_1+\dots+a_it_i+a_{i+1}t_{i+1}+ \dots +a_nt_n)
\end{equation*}
when $h=\diag{0,\seq a, \seq{-a}}$. Thus, $t_1$, \dots, $t_n$ are such that
\begin{equation*}
  \textstyle  a_1t_1+\dots+a_{n-1} t_{n-1} +a_n\big(t_n-\frac 1{4n-2}\big)=0 
\end{equation*}
for all $a_1$, \dots, $a_n$ in $\BBC$. Taking $a_j=1$ and $a_k=0$ for $k\ne
j$ we see that
\[
t_j= \begin{cases} \textstyle \frac 1{4n-2}&j=n \\ 0 &j\ne n. \end{cases}
\]
Therefore, $t_{\alpha_n}=\frac 1{4n-2} d_n$.

%%% Local Variables: 
%%% mode: latex
%%% TeX-master: "mydissertation"
%%% End: 
